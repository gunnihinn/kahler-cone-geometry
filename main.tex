\documentclass[11pt,a4paper]{amsart}
\usepackage[english]{babel}
\usepackage[T1]{fontenc}
\usepackage[utf8]{inputenc}
\usepackage{lmodern}
\usepackage{amsmath,amsfonts,amssymb}
\usepackage[pdftex]{hyperref}
\usepackage{setspace}
\setstretch{1.1}
\usepackage[usenames,dvipsnames]{xcolor}
\usepackage{tikz}

\title{Curvature of K\"{a}hler moduli}
\author{Gunnar Þór Magnússon}

\newtheorem{theo}{Theorem}[section]
\newtheorem{prop}[theo]{Proposition}
\newtheorem{coro}[theo]{Corollary}
\newtheorem{lemm}[theo]{Lemma}
\newtheorem*{theo*}{Theorem}
\theoremstyle{definition}
\newtheorem*{defi}{Definition}
\newtheorem{exam}[theo]{Example}
\theoremstyle{remark}
\newtheorem*{rema}{Remark}

\newcommand{\QQ}{\mathbb{Q}}
\newcommand{\RR}{\mathbb{R}}
\newcommand{\ZZ}{\mathbb{Z}}
\newcommand{\CC}{\mathbb{C}}
\newcommand{\PP}{\mathbb{P}}

\newcommand{\End}{\mathop{\mathrm{End}}}
\newcommand{\Aut}{\mathop{\mathrm{Aut}}}
\newcommand{\id}{\mathop{\mathrm{id}}}
\newcommand{\Vol}{\mathop{\mathrm{Vol}}}
\def\Im{\mathop{\rm Im}}

\def\ov#1{\overline{#1}}

\def\half{\tfrac12}
\def\onfo{\tfrac14}

\def\E#1{\coho{#1,#1}}

%% Operators
\DeclareMathOperator{\Gr}{Gr}
\DeclareMathOperator{\Hom}{Hom}
\def\d{\partial}
\def\dbar{\bar\partial}

\def\p{\times}

%% Inner product
\def\llangle{\langle\!\langle}
\def\rrangle{\rangle\!\rangle}

%% Cohomology groups
\def\coho#1{\mathrm{H}^{#1}}

%% Connection and curvature
\def\levi{\nabla}
\newcommand{\levicivita}[1]{
d#1
-\tfrac12 g(d\kf, \kf) \cdot #1
-\tfrac12 g(#1, \kf) \cdot d\kf
+\tfrac12 \Lambda(#1 \wedge d\kf)
}
\newcommand{\Levicivita}[2]{
d_{#1}#2
-\tfrac12 g(#1, \kf) \cdot #2
-\tfrac12 g(#2, \kf) \cdot #1
+\tfrac12 \Lambda(#2 \wedge  #1)
}
\def\chern{D}
\def\conn{\nabla}
\def\curv{\tfrac{i}{2\pi}\Theta}

%% Intersection products with a Kahler form
\def\q#1{\frac{1}{\Vol}\int_X #1 \wedge \kf\^{n-1}}
\def\qq#1#2{\frac{1}{\Vol}\int_X #1 \wedge #2 \wedge \kf\^{n-2}}
\def\qqq#1#2#3{\frac{1}{\Vol}\int_X #1 \wedge #2 \wedge #3 \wedge \kf\^{n-3}}
\def\qqqq#1#2#3#4{\frac{1}{\Vol}\int_X #1 \wedge #2 \wedge#3 \wedge#4 \wedge \kf\^{n-4}}

%% The K\"{a}hler form and class of a Hermitian metric
\def\kf{\omega}
\def\ckf{\alpha}
\def\Lef{\Lambda}

%% Vertical and horizontal vectors
\def\ver#1{#1^{\mathrm{V}}}
\def\hor#1{#1^{\mathrm{H}}}

%% Tangent vectors in K\"{a}hler cone
\def\ton{u}
\def\ttw{v}
\def\tth{z}
\def\tfo{w}

%% Algebra power
\def\^#1{^{[#1]}}

%% K\"{a}hler cones
\def\KC{C}
\def\CKC{\mathbb{C}C}
\def\RKC{\mathcal{\KC}}

\begin{document}

\begin{abstract}
The K\"{a}hler cone of a compact Kahler manifold carries a natural
Riemannian metric, given by the intersection product of its cohomology
ring. We give cohomological expressions for the geodesics, Levi-Civita
connection and curvature tensor of this metric, show that any two points
in the cone can be joined by a geodesic, and determine when the metric
is complete.
\end{abstract}

\maketitle



\section*{Introduction}

Let $X$ be a compact K\"{a}hler manifold of complex dimension $n$. The
Kahler cone of $X$ is the set of Kahler classes, that is,
$(1,1)$-classes that contain a Kahler metric. Each Kahler class defines
an inner product on the space of $(1,1)$-classes and letting the classes
vary defines a natural Riemannian metric on the Kahler cone. 

This metric has been studied by Wilson~\cite{Wilson}, Wilson and
Trenner~\cite{WilsonTrenner} and myself~\cite{Magnusson}, mostly by
embedding the Kahler cone into the space of smooth Hermitian metrics on
the manifold via the Aubin--Calabi--Yau theorem and using the natural
$L^2$ metric defined on that infinite-dimensional space. See also
Huybrechts~\cite{Huybrechts} for another similar approach to the Kahler
cone. 

Working in this fashion one can calculate the curvature tensor of
the metric, but the expression this yields does not descend again to the
level of cohomology, most fundamentally because the wedge product of
harmonic forms is generally not harmonic.  The resulting expressions are
also a bit hard to work with, for example if one wants to calculate any
of the derived curvatures of the metric or try to estimate the curvature
tensor. This is reflected in that the most complete and interesting
results that have been obtained on the curvature this metric have in
fact been proven by working in the cohomology ring in special cases like
on complex tori, compact surfaces or Calabi--Yau threefolds; see
\cite{Wilson,WilsonTrenner}.

In this paper we treat the Aubin--Calabi--Yau theorem and actual Kahler
metrics as red herrings and work entirely within the cohomology ring of
our manifold. This shifts the difficulty of the problem from dealing
with wild differential forms to calculating various cup products in the
cohomology ring. Those products turn out to be much more manageable than
the differential forms and yield very satisfying results. For example,
we're able to give explicit formulas for the geodesics of the metric
(Theorem~\ref{theo:geodesics}) and say when it is complete, calculate
that its
Levi-Civita connection is
$$
\conn_{\ton} \ttw
= d_{\ton} \ttw 
- \half \Lef(\ton) \ttw 
- \half \Lef(\ttw) \ton
+ \half \Lef(\ton \cup \ttw)
$$
(see Proposition~\ref{prop:connection}) and calculate that the curvature
tensor is
$$
R(\ton,\ttw,\tth,\tfo)
=
- \onfo \langle \Lef(\ton \cup \tfo), \Lef(\ttw \cup \tth) \rangle
+ \onfo \langle \Lef(\ton \cup \tth), \Lef(\ttw \cup \tfo) \rangle
$$
(see Theorem~\ref{theo:curvature} and the remark following it for a
natural way to express the curvature tensor as a perturbation of a
tensor of constant negative sectional curvature).  These formulas show
very clearly why the differential form approaches do not give totally
satisfying results, because the curvature involves cup products of
$(1,1)$-classes where attempts to represent them with harmonic forms
fail.


\section{The K\"{a}hler cone}

Let $X$ be a compact K\"{a}hler manifold of dimension $\dim_{\CC} X = n$.

\begin{defi}
The \emph{K\"{a}hler cone} of $X$ is the set
\begin{equation*}
C(X) = \{ \kf \in \coho{1,1}(X,\RR) 
\mid
\text{$\kf$ contains a K\"{a}hler metric}
\}.
\end{equation*}
\end{defi}

If there can be no confusion about the underlying manifold $X$, we'll
just write $\KC$ for its K\"{a}hler cone.  As the name suggests, this is
an open cones in the finite-dimensional vector space
$\coho{1,1}(X,\RR)$. The K\"{a}hler cone is the trancendental analogue
of the ample cone of a projective variety. It is described by the the
trancendental version of the Nakai--Moishezon criteria due to Demailly
and Paun~\cite{DemaillyPaun}:


\begin{theo}
The K\"{a}hler cone of $X$ is a connected component of the set of real
$(1,1)$-cohomology classes that are numerically positive on analytic
cycles, that is, classes $\alpha$ such that $\int_{Z} a^p > 0$ for every
irreducible analytic set $Z$ in $X$ of dimension $p$.
\end{theo}



Since the K\"{a}hler cone $\KC$ is an open set in a real vector space,
we can view it as a smooth manifold in its own right. It is in fact
naturally a Riemannian manifold, because each Kahler class defines an
inner product on the Kahler cone, and these inner products vary smoothly
with the underlying class.


Let's agree on some notation before we find conventient expressions for
this metric. If $x$ is an element of the cohomology ring of $X$, we
write $x\^k := x^k/k!$ for all $k \geq 0$. This notation is
quite convenient for calculations with Kahler forms in the cohomology
ring of $X$; I learned it from Georg Schumacher.


\begin{prop}
Let $\ton,\ttw$ be elements of $T_{\kf}C$. The Riemannian metric on $C$
can be defined in any of the following ways.
\begin{enumerate}
    \item
\hfil
$
\langle \ton, \ttw \rangle
= \Lef(\ton)\Lef(\ov\ttw)
- \Lef\^2(\ton\wedge\ov\ttw).
$
\hfil
    \item 
$
\hfil
\langle\ton, \ttw \rangle
= 
\frac{1}{\Vol} \langle\!\langle \hat\ton, \hat\ttw \rangle\!\rangle,
%\frac{1}{\Vol} \int_X \hat\ton \wedge {\hat\kf}\^{n-1}
%\frac{1}{\Vol} \int_X \hat\ttw \wedge {\hat\kf}\^{n-1}
%- \frac{1}{\Vol} \int_X \hat\ton \cup \hat\ttw \wedge {\hat\kf}\^{n-2},
\hfil
$

\noindent
where the right-hand side is the $L^2$ inner product defined by a Kahler
metric in $\kf$ and $\hat\ton,\hat\ttw$ are the are the harmonic
representatives of $\ton, \ttw$.

    \item
The metric is the quadradic form defined by the Hessian of $-\Vol$.
\end{enumerate}
\end{prop}


\begin{proof}
That $(1)$ and $(2)$ define the same quadradic form is a simple
calculation based on the adjoint property of $\Lef$, and that $(1)$
agrees with the inner product that $\kf$ defines can be seen by taking
the primitive decomposition of $\ton$ and $\ttw$, plugging it into $(1)$
and calculating until the Hodge--Riemann bilinear relations say that we
have the correct inner product.

Note that we can view $\kf$ as the tautological section $C \to T_C$
associated to the tangent bundle of any open set in a vector space.
Then $d_{\ttw} \Vol = -\Lef(\ttw)$ as we see by considering
$\Lef(\ttw) \kf\^n = \ttw \cup \kf\^{n-1}$. Similarly, we find that
$\operatorname{Hess}(\ton,\ttw) \Vol 
= d_{\ton} d_{\ttw}\Vol = \langle \ton, \ttw \rangle$ by comparing with
$(1)$.
\end{proof}





\section{Geodesics and completeness}


\begin{theo}
\label{theo:geodesics}
Let $\kf$ be a point in $C$ and let $x \in T_{\kf} C$. Decompose $x =
\Lef(x) \, \kf + u$, where $u = x - \frac{1}{n}\Lef(x)\,\kf$ is
primitive with respect to $\kf$.  The geodesic $\gamma$ that satisfies
$\gamma(0) = \kf$ and $\gamma'(0) = x$ is
$$
\gamma(t)
= \Vol(\kf)^{1/n} e^{\Lef(x)t} \frac{\kf + t u}{\Vol(\kf + t u)^{1/n}}
$$
for $|t|$ small enough so that $\kf + tu \in C$.
\end{theo}


\begin{proof}
We invite the reader to check that $\gamma(0) = \kf$ and $\gamma'(0) =
x$. Now consider the splitting $C \to \RR \times C_1$, $x \mapsto
(\log \Vol(x), x/\Vol(x)^{1/n})$, where $C_1$ is the set of Kahler classes of
volume one. This splitting is an isometry if we give $\RR$ its usual
flat metric and $C_1$ the metric induced by restriction. Our $\gamma$
maps to $(\frac{1}{n}\log \Vol(\kf) + \Lef(x) t, (\kf + t u)/\Vol(\kf + t u)^{1/n})$ under
the splitting, so we can check that each factor is a geodesic
separately. The factor in $\RR$ clearly is so.

Consider thus the two ordinary differential equations
\begin{align}
\label{eq:geodesic}
\conn_{\gamma'(t)} \gamma'(t) &= 0,
\qquad \gamma(0) = \kf,\quad \gamma'(t) = u
\\
\label{eq:fixedvol}
\gamma'(t) \cup \gamma(t)\^{n-1} &= 0,
\qquad \gamma(0) = \kf,\quad \gamma'(t) = u
\end{align}
where we assume $\kf$ has volume one and $u$ is primitive with respect
to $\kf$. The first is of course the geodesic equation in $C_1$. The
second is obtained by differentiating the equation $\Vol(\gamma(t)) = 1$
and says that $\gamma'(t)$ should always be primitive with respect to
$\gamma(t)$.

We now claim that any geodesic of $C_1$ satisfies \eqref{eq:fixedvol}.
This is clear, because if $\gamma(t)$ is a geodesic in $C_1$, then
$\gamma'(t) \in T_{\gamma(t)} C_1$ for all $t$, and the tangent space of
$C_1$ at $\gamma(t)$ is exactly the set of classes that are primitive
with respect to $\gamma(t)$. By the uniqueness of solutions to ordinary
differential equtions, the equations \eqref{eq:geodesic} and
\eqref{eq:fixedvol} thus have the same solutions.

Now, given $u \in T_{\kf} C_1$ we consider the curve $\gamma(t) = (\kf +
tu)/v(t)^{1/n}$, where we write $v(t) = \Vol(\kf + tu)$ for short. It
is in $C_1$ for all small enough $t$ by construction. We have
$$
v(t)
= \sum_{k=0}^n \Lef\^{k}(u^k) \, t\^k
= 1 + \Lef\^{2}(u^2) \, \tfrac12 t^2 + O(t^3),
$$
where $\Lef$ is the adjoint Lefschetz operator associated to $\kf$. Thus
$v(0)=1$ and $v'(0)=0$. Then
$$
\gamma'(t) = \frac{u}{v(t)^{1/n}} - \frac{v'(t)}{n v(t)} \gamma(t)
$$
so $\gamma(0) = \kf$ and $\gamma'(0) = u$. The curve $\gamma$ is thus a
solution of \eqref{eq:fixedvol}, which means that it is also the geodesic 
that passes through $\kf$ in the direction of $u$. We finally remark
that $\gamma(t)$ is defined for all $t$ such that the volume of $\kf +
tu$ is positive.
\end{proof}


\begin{coro}
Any two points in $C$ can be joined by a geodesic. 
\end{coro}



\begin{coro}
The geodesic completion of the Kahler cone is a connected component of
the cone of classes of positive volume.
\end{coro}

\begin{proof}
Let $\kf$ be a point in $C$. Given any vector $x \in T_{\kf}C$ we can
construct a geodesic $\gamma$ that departs from $\kf$ in the direction
of $x$ and is defined as long as $\Vol(\gamma)$ is neither $0$ nor
$\infty$. Given any such class, we can also explicitly construct a
geodesic that starts at $\kf$ and tends towards it.
\end{proof}


\subsection*{Pullbacks}

A holomorphic map $f : X \to Y$ between compact K\"{a}hler manifolds induces
a morphism $f^* : \coho{*}(Y,\RR) \to \coho{*}(X,\RR)$ in cohomology
that respects the Hodge decomposition. However, if $\kf$ is a K\"{a}hler
class on $Y$, then $f^*\kf$ is hardly ever a K\"{a}hler class on $X$. This
happens mostly if $f$ is either an embedding or a finite covering map.

\begin{prop}
Let $f : X \to Y$ be a finite surjective morphism. Let $g_X$ and $g_Y$
be the metrics on the K\"{a}hler cones of $X$ and $Y$, respectively.
Then the pullback morphism $f^* : \KC(Y) \to \KC(X)$ is a Riemannian
embedding.
\end{prop}

\begin{proof}
Let $\ckf$ be a point in $\KC(Y)$ and $\kf = \Im \ckf$. The volume of
$X$ with respect to $f^*\kf$ is
\begin{equation*}
  \Vol(X,f^*\kf) = p \, \Vol(Y,\kf)
\end{equation*}
as $f$ is finite of degree $p$. It follows that $f^*$ is an embedding.
\end{proof}

\begin{coro}
The group $\Aut X$ of holomorphic automorphisms of $X$ acts by
isometries on the K\"{a}hler cone $\KC(X)$.
\end{coro}

A closer look reveals that this last statement contains less information
than first meets the eye. The automorphism group $\Aut X$ of a compact
complex manifold is a Lie group and it splits roughly into two parts; a
positive-dimensional group given by the flows of holomorphic vector
fields, or elements of $\coho{0}(X,T_X)$, and a discrete part consisting of
``other'' automorphisms. The isomorphisms generated by vector fields act
trivially on the cohomology ring of $X$, so the only part of $\Aut X$
that possibly acts by nontrivial isometries on $\KC(X)$ is discrete.





\section{The Levi-Civita connection}

We start with a couple of preliminary computations.

\begin{lemm}
If $u_1, \ldots, u_k$ are real $(1,1)$-classes, then
$$
\displaylines{
d_{\ttw} \Lef\^{k}(u_1 \cup \cdots \cup u_k)
= -\Lef(\ttw) \Lef\^{k}(u_1 \cup \cdots \cup u_k)
\hfill\cr\hfill{}
+ \Lef\^{k}(d_{\ttw} u_1 \cup \cdots \cup u_k)
+ \cdots +
\Lef\^{k}(u_1 \cup \cdots \cup d_{\ttw} u_k)
\hfill\cr\hfill{}
+ \Lef\^{k+1}(u_1 \cup \cdots \cup u_k \cup \ttw).
}
$$
\end{lemm}

\begin{proof}
This is clear once we write
$$
\Lef\^{k}(u_1 \cup \cdots \cup u_k)
= \frac{1}{\Vol(X,\kf)} \int_X u_1 \cup \cdots \cup u_k \cup \kf\^{n-k}
$$
and compute.
\end{proof}

\begin{lemm}
\label{lemm:triple}
Let $\ton,\ttw,\tth$ be $(1,1)$-classes. Then
$$
\langle \Lef(\ton\cup\ttw), \tth \rangle
= - \Lef\^3(\ton \cup \ttw \cup \tth)
+ \Lef\^2(\ton \cup \ttw) \Lef(\tth).
$$
\end{lemm}


\begin{proof}
First note that if $\tth$ is a $(1,1)$-class, then $\tth = (\tth -
\frac1n \Lef(\tth)\kf) + \frac1n \Lef(\tth) \kf$ is its primitive
decomposition. Then
\begin{align*}
*(\kf \cup \tth)
&= *( \kf \cup (\tth - \tfrac1n \Lef(\tth)\kf))
+ * \Bigl( \tfrac{2}{n} \Lef(\tth) \kf\^2 \bigr)
\\
&= -(\tth - \tfrac1n \Lef(\tth)\kf) \cup \kf\^{n-3}
+ \tfrac{2}{n} \Lef(\tth) \kf\^{n-2}
\\
&= -\tth \cup \kf\^{n-3}
+ \tfrac{n-2}{n} \Lef(\tth) \kf\^{n-2}
+ \tfrac{2}{n} \Lef(\tth) \kf\^{n-2}
\\
&= -\tth \cup \kf\^{n-3}
+ \Lef(\tth)\, \kf\^{n-2}.
\end{align*}
We now get
\begin{align*}
\langle \Lef(\ton \cup \ttw), \tth \rangle \,\kf\^n
&= \Lef(\ton \cup \ttw) \cup 
(-\tth \cup \kf\^{n-2} + \Lef(\tth)\, \kf\^{n-1})
\\
&= - \Lef\^2(\Lef(\ton \cup \ttw) \cup \tth) \,\kf\^{n}
+ 2 \Lef\^2(\ton \cup \ttw) \Lef(\tth) \,\kf\^{n},
\end{align*}
which proves the result.
\end{proof}


Recall that the Levi-Civita connection is the unique connection on the tangent
bundle that's compatible with the metric and is torsion-free. That is, it
satisfies
$$
d g(\ton, \ttw) 
= g(\levi \ton, \ttw) + g(\ton, \levi \ttw),
\quad
\levi_{\ton}\ttw - \levi_{\ttw}\ton = [\ton,\ttw]
$$
for all sections $\ton, \ttw$ of the bundle.


\begin{prop}
\label{prop:connection}
The Levi-Civita connection of the Riemannian metric $g$ on $\KC$ is
$$
\levi_{\tth} \ton
=
d_u \tth
-\tfrac12 \Lef(\ton) \tth
-\tfrac12 \Lef(\tth) \ton
+\tfrac12 \Lef(\ton \cup \tth).
$$
\end{prop}

\begin{proof}
The connection we've written down satisfies $\conn_{\ton} \ttw - \conn_{\ttw}
\ton = [\ton, \ttw]$ by inspection. We turn to its computation.

The metric is defined by
$$
\langle \ton, \ttw \rangle
= \Lef \ton \, \Lef \ttw
- \Lef\^2(\ton \cup \ttw).
$$
Taking the derivative of this in the $\tth$ direction gives
$$
\displaylines{
d_{\tth} \langle \ton, \ttw \rangle
= 
- \Lef(\ton) \Lef(\ttw) \Lef(\tth)
+ \Lef(d_{\tth}\ton) \Lef(\ttw)
+ \Lef\^{2}(\ton \cup \tth) \Lef(\ttw)
\hfill\cr\hfill{}
- \Lef(\ton) \Lef(\ttw) \Lef(\tth)
+ \Lef(\ton) \Lef(d_{\tth}\ttw)
+ \Lef(\ton) \Lef\^{2}(\ttw \cup \tth)
\cr\hfill{}
+ \Lef(\tth) \Lef\^2(\ton \cup \ttw)
- \Lef\^2(d_{\tth} \ton \cup \ttw)
- \Lef\^2(\ton \cup d_{\tth} \ttw)
- \Lef\^3(\ton \cup \ttw \cup \tth)
\cr{}
\phantom{d_{\tth} \langle \ton, \ttw \rangle}
=: \langle d_{\tth} \ton, \ttw \rangle
+ \langle \ton, d_{\tth} \ttw \rangle
+ A(\ton, \ttw, \tth).
\hfill
}
$$
We're going to write $A = \frac12 A + \frac12 A$ and try to write the first half
as an inner product with $\ttw$, and the second half as an inner product with
$\ton$.

To that end, we note that
$$
\displaylines{
A(\ton, \ttw, \tth)
= 
- \Lef(\ton) \Lef(\ttw) \Lef(\tth)
+ \Lef\^{2}(\ton \cup \tth) \Lef(\ttw)
\hfill\cr\hfill{}
- \Lef(\ton) \Lef(\ttw) \Lef(\tth)
+ \Lef(\ton) \Lef\^{2}(\ttw \cup \tth)
\cr\hfill{}
+ \Lef(\tth) \Lef\^2(\ton \cup \ttw)
- \Lef\^3(\ton \cup \ttw \cup \tth)
\cr{}
\phantom{A(\ton, \ttw, \tth)}
= 
- \Lef(\tth) \langle \ton, \ttw \rangle
- \Lef(\ton) \langle \tth, \ttw \rangle
+ \Lef\^{2}(\ton \cup \tth) \Lef(\ttw)
- \Lef\^3(\ton \cup \ttw \cup \tth)
\hfill\cr\hfill{}
=
- \langle \Lef(\tth) \ton, \ttw \rangle
- \langle \Lef(\ton) \tth, \ttw \rangle
+ \langle \Lef(\ton \cup \tth), \ttw \rangle
}
$$
by Lemma~\ref{lemm:triple}, which can indeed be written as an inner product with
$\ttw$. Since $A$ is symmetric in $\ton, \ttw, \tth$, we can start again from
$A$ and write it as an inner product with $\ton$. Taking half of each, we arrive
at our claimed form of the connection.
\end{proof}


\begin{coro}
\label{coro:kahlerform}
\begin{itemize}
    \item 
$\levi \kf = 0$.
    \item 
If $\ton$ is a primitive vector field, then $\levi \ton$ is also
primitive.
\end{itemize}
\end{coro}






\section{The curvature tensor}



\begin{theo}
\label{theo:curvature}
The curvature tensor of the metric on the Kahler cone is
\begin{equation*}
R(\ton,\ttw,\tth,\tfo)
= 
- \onfo \langle \Lef(\ton \cup \tfo), \Lef(\ttw \cup \tth) \rangle
+ \onfo \langle \Lef(\ton \cup \tth), \Lef(\ttw \cup \tfo) \rangle.
\end{equation*}
\end{theo}


\begin{proof}
We may assume that all the tangent fields $\ton,\ttw,\tth,\tfo$ are
primitive, either by appealing to the isometric splitting of the Kahler
cone, by using that $\conn \kf = 0$ and the symmetries of the curvature
tensor to see that $R$ always degenerates to the primitive parts of our
classes, or by calculating the curvature tensor first for primitive
classes and then doing painful algebra to see that the general case
degenerates to that one. However we do it, we find that for primitive
fields we have
$$
\conn_{\ttw} \tth
= d_{\ttw} \tth + \half \Lef(\ttw \cup \tth)
$$
and
$$
\displaylines{
\conn_{\ton} \conn_{\ttw} \tth
= d_{\ton} d_{\ttw} \tth 
+ \half (d_{\ton} \Lef)(\ttw \cup \tth)
+ \half \Lef(d_{\ton} \ttw \cup \tth)
\hfill\cr\hfill{}
+ \half \Lef(\ttw \cup d_{\ton} \tth)
+ \half \Lef(\ton \cup d_{\ttw} \tth)
+ \onfo \Lef(\ton \cup \Lef(\ttw \cup \tth)).
}
$$
This gives
$$
R(\ton,\ttw) \tth
= 
\half (d_{\ton} \Lef)(\ttw \cup \tth)
+ \onfo \Lef(\ton \cup \Lef(\ttw \cup \tth))
- \half (d_{\ttw} \Lef)(\ton \cup \tth)
- \onfo \Lef(\ttw \cup \Lef(\ton \cup \tth))
$$
since the other terms either make up $\conn_{[\ton,\ttw]}\tth$ or are
symmetric in $\ton,\ttw$. To make sense of this, it's convenient to take
the inner product with $\tfo$.

We have $*(\kf \cup \tfo) = -\kf\^{n-3} \cup \tfo$ since $\tfo$ is
primitive, so 
$$
\langle \Lef(\ttw \cup \tth), \tfo \rangle
= \langle \ttw \cup \tth, \kf \cup \tfo \rangle
= -\Lef\^3(\ttw \cup \tth \cup \tfo).
$$
Differentiating this in the direction of $\ton$ gives
$$
\displaylines{
\langle (d_{\ton}\Lef)(\ttw \cup \tth), \tfo \rangle
+ \half \langle \Lef(\ton \cup \Lef(\ttw \cup \tth)), \tfo \rangle
+ \half \langle \Lef(\ttw \cup \tth), \Lef(\ton \cup \tfo) \rangle
\hfill\cr\hfill{}
= 
%-\Lef(\ton) \Lef\^3(\ttw \cup \tth \cup \tfo)
-\Lef\^4(\ton \cup \ttw \cup \tth \cup \tfo)
}
$$
after canceling out the terms that involve the derivatives of the
tangent fields. Then one part of the curvature tensor is
$$
\displaylines{
\half \langle (d_{\ton}\Lef)(\ttw \cup \tth), \tfo \rangle
+ \onfo \langle \Lef(\ton \cup \Lef(\ttw \cup \tth)), \tfo \rangle
\hfill\cr\hfill{}
%- \half \Lef\^4(\ton \cup \ttw \cup \tth \cup \tfo)
%- \onfo \langle \Lef(\ton \cup \Lef(\ttw \cup \tth)), \tfo \rangle
%\cr\hfill{}
%- \onfo \langle \Lef(\ttw \cup \tth), \Lef(\ton \cup \tfo) \rangle
%+ \onfo \langle \Lef(\ton \cup \Lef(\ttw \cup \tth)), \tfo \rangle
%\cr{}
=- \half \Lef\^4(\ton \cup \ttw \cup \tth \cup \tfo)
- \onfo \langle \Lef(\ttw \cup \tth), \Lef(\ton \cup \tfo) \rangle.
}
$$
The first term is symmetric in $\ton,\ttw$, so we get
$$
R(\ton,\ttw,\tth,\tfo)
= 
- \onfo \langle \Lef(\ton \cup \tfo), \Lef(\ttw \cup \tth) \rangle
+ \onfo \langle \Lef(\ton \cup \tth), \Lef(\ttw \cup \tfo) \rangle
$$
as promised.
\end{proof}



The metric has constant sectional curvature $-k$ if and only if 
its curvature tensor is equal to
$$
\displaylines{
R(\ton,\ttw,\tth,\tfo)
= 
- k\langle \ton, \tfo \rangle \langle \ttw, \tth \rangle
+ k \langle \ton, \tth \rangle \langle \ttw, \tfo \rangle.
}
$$
This is satisfied for our metric if and only if
$$
\displaylines{
- k \langle \ton \cup \tfo, \ttw \cup \tth \rangle
+ k \langle \ton \cup \tth, \ttw \cup \tfo \rangle,
\hfill\cr\hfill{}
 =
- (4k+1) \langle \ton, \tfo \rangle \langle \ttw, \tth \rangle
+ (4k+1) \langle \ton, \tth \rangle \langle \ttw, \tfo \rangle
}
$$
for all $\ton,\ttw,\tth,\tfo$ and all Kahler classes $\kf$. Suppose we
now pick unit vectors $\ttw = \tth$ and $\ttw \perp \tfo$. Then
$$
\displaylines{
- k \langle \ton \cup \tfo, \ttw \cup \ttw \rangle
+ k \langle \ton \cup \ttw, \ttw \cup \tfo \rangle,
 =
- (4k+1) \langle \ton, \tfo \rangle
}
$$

Alternatively, this holds if and only if
$$
\displaylines{
- \onfo \langle \Lef(\ton \cup \tfo), \Lef(\ttw \cup \tth) \rangle
+ \onfo \langle \Lef(\ton \cup \tth), \Lef(\ttw \cup \tfo) \rangle
\hfill\cr\hfill{}
 =
- k \langle \ton, \tfo \rangle \langle \ttw, \tth \rangle
+ k \langle \ton, \tth \rangle \langle \ttw, \tfo \rangle.
}
$$
I wonder what happens if we take the derivative of that equation?

Basically, I'd like to show the metric almost never has constant sectional
curvature. At least that would be something new. It would be nice if I
could first show that to have constant sectional curvature we must have
$$
\langle \ton \cup \tfo, \ttw \cup \tth \rangle
= \lambda \langle \ton, \tfo \rangle \langle \ttw, \tth \rangle
$$
for some $\lambda \geq 0$, all tangent vectors and all Kahler classes.
Ideally, that would force us to either have $n = 2$ (if $\lambda \not=
0$) or $h^{1,1} = 1$ (else). For the second we'd have $\kf^2 = 0$, so $n
= 1$. For the first we'd get $\ton = 0$ for any primitive $\ton$ by
evaluating at $(\ton,\kf,\kf,\kf)$, so $h^{1,1} = 1$.



\begin{rema}
If $x,y$ are $(2,2)$-classes, then $\Lef\^4(x \cup y) = \langle x, y
\rangle - \langle \Lef(x), \Lef(y) \rangle + \langle \Lef\^2(x),
\Lef\^2(y) \rangle$. An alternate expression for the curvature tensor is
thus
\begin{equation*}
\displaylines{
R(\ton,\ttw,\tth,\tfo)
= 
- \onfo \langle \ton, \tfo \rangle 
    \langle \ttw, \tth \rangle
+ \onfo \langle \ton, \tth \rangle 
    \langle \ttw, \tfo \rangle
\hfill\cr\hfill{}
- \onfo \langle \ton \cup \tfo, \ttw \cup \tth \rangle
+ \onfo \langle \ton \cup \tth, \ttw \cup \tfo \rangle,
}
\end{equation*}
so the curvature tensor is a perturbation of the curvature tensor of a
space form of constant sectional curvature $-1/4$. Unfortunately there
is no known way to control the perturbation terms in general; bounding them
from above is impossible by Wilson and Trenner's
example~\cite{WilsonTrenner}, and bounding them from below is beyond
our reach in the cohomology ring.

It is sometimes possible to bound these curvatures, for example in the
cohomology ring of a complex torus or on a compact Kahler surface. Such
bounds seem to be related to the optimal real numbers $M(\kf)$ such that
$$
| \ton \cup \ttw |^2 \leq M(\kf)\, |\ton|^2 |\ttw|^2
$$
for all $\ton, \ttw \in \coho{1,1}(X,\RR)$.  On a
surface we have $M(\kf) = 1$ and on a complex torus $M(\kf) = 2(n-1)/n$
for all $\kf$. In general, the function $\kf \mapsto M(\kf)$ is lower
semicontinuous and $M(\kf) \geq 2(n-1)/n$; Wilson and Trenner's example
then shows that $M(\kf)$ indeed varies with $\kf$ and can tend to
infinity. If $M(\kf)$ can be bounded uniformly in $\kf$, then the
curvatures of our metric are bounded as well. 

I don't know of a condition on $X$ that forces the $M(\kf)$ to
be bounded. By comparison with the completeness of the metric it would
be tempting to think it related to subvarieties whose volume can be
contracted to zero without contracting the whole ambient variety, and
that is what happens in Wilson and Trenner's example where the curvature
blows up, but that clearly doesn't hold because the $M(\kf)$ are
uniformly bounded on surfaces where we can blow up points and contract
the volume of the exceptional divisors.
\end{rema}




\bibliographystyle{alpha}
\bibliography{main}

\end{document}
